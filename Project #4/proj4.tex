\documentclass[a4paper, 11pt]{article}

%     ITY 2020/2021 Project #4     %
% By Aleksandr Verevkin (xverev00) %

\usepackage[czech]{babel}
\usepackage[utf8]{inputenc}
\usepackage[left=2cm, top=3cm, text={17cm, 24cm}]{geometry}
\usepackage{times}
\usepackage[unicode]{hyperref}

\begin{document}

\begin{titlepage}

\begin{center}
\Huge
\textsc{Vysoké učení technické v~Brně}\\
\huge
\textsc{Fakulta informačních technologií}\\
\vspace{\stretch{0.382}}
\LARGE
Typografie a publikování\,--\,4. projekt\\
\Huge
Bibliografické citace\\
\vspace{\stretch{0.618}}
\end{center}
{\Large \today \hfill Aleksandr Verevkin}

\end{titlepage}

\section{Úvod}
Typografie je nauka zabývající se písmem, jeho výběrem, použitím a sazbou. 
Je také souborem pravidel, které napomáhají dosažení cíle.
Někteří označují typografii nikoli pouze za nauku či metodu, ale také za umění, jak správně 
pracovat s~jednotlivými typy fontů, velikostí písma, řádkování, rozestupy mezi jednotlivými 
znaky, či uspořádaní písma. Článek \cite{TypografieArticle} uvádí o~vizualizace dat pomocí typografie.

\section{Úvod do \TeX u}
\TeX\ je značkovací jazyk vytvořený Donaldem Knuthem pro atraktivní a konzistentní sazbu dokumentů.
Je to také programovací jazyk v~tom smyslu, že podporuje konstrukci if-else, dokáže vypočítat atd.
Díky ovládacímu prvku \TeX\ je velmi výkonný, ale také obtížný a časově náročný na použití,
viz \cite{book2006}.

\LaTeX\ představuje nadstavbu \TeX u o~balíčky předdefinovaných maker, 
která usnadňují a zpřístupňují složitý jazyk \TeX u pro sazbu dokumentů širšímu spektru uživatelů.
O~tom co je \TeX\ a co není pojednává \cite{onlineFEKT}.

\section{Editory, nástroje}
\begin{itemize}
    \item Kile, Vim, Emacs(Linux)
    \item WinEdt, WinTeX, Ultra Edit(Windows)
    \item Overleaf(Online)
    \item Více o~editorech je \cite{onlineFIT}.
\end{itemize}
O~převod dokumentů z~aplikace Word do \LaTeX u věnuje \cite{WorkSimek}.

\section{Základy \LaTeX ového souboru}
\LaTeX ový soubor obsahuje zdrojový text, který má být zpracován k~vytvoření tištěného výstupu.
Rozdělení textu na řádky stejné šířky, formátování do odstavců a rozdělení na stránky s~čísly 
stránek a běžícími hlavami jsou všechny funkce zpracovatelského programu, 
nikoli samotného vstupního textu, viz \cite{book2003}.

\section{Fonty}
Plain\TeX\ má načtenu rodinu fontů Computer Modern a CSplain má zavedenu víceméně
stejnou rodinu fontů zvanou CSfonty. Ta rozšiřuje tabulku znaků Computer Modern fontů
o~znaky české a slovenské abecedy. 
Chcete-li změnit typ písma dokumentu, je třeba přidat jednoduchý řádek:
\verb|\usepackage{font}|, 
typy písma lze najít \cite{OnlineOverleaf}.

Více o~fontech \cite{FullArticle}.

\section{Sazba matematických vzorců}
Základními matematickými prostředími jsou řádkové rovnice a samostatné rovnice. 
Od textu je odděluje symbol \$ v~případě řádkového prostředí, 
resp. \verb|\[|\dots\verb|\]| v~případě samostatné rovnice. 
Alternativně se často používá k~oddělování matematických symbolů \verb|\(|\dots\verb|\)|, \$\$ \dots \$\$. 
Více o~matematické sazbě \cite{WorkHolan}.

Existují i zařízení pro převod matematických textů z~ručně psaných textů \cite{MathArticle}.

\pagebreak

\bibliographystyle{czechiso}
\renewcommand{\refname}{Literatura}
\bibliography{proj4}

\end{document}
