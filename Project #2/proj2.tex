\documentclass[a4paper, 11pt, twocolumn]{article}

\usepackage[czech]{babel}
\usepackage[utf8]{inputenc}
\usepackage[IL2]{fontenc}
\usepackage[left=1.5cm, top=2.5cm, text={18cm, 25cm}]{geometry}
\usepackage{times}
\usepackage{amsthm, amssymb, amsmath}
\usepackage[unicode]{hyperref}

\theoremstyle{definition}
\newtheorem{definition}{Definice}
\theoremstyle{plain}
\newtheorem{theorem}{Věta}

\begin{document}

\begin{titlepage}

\begin{center}
\Huge
\textsc{Fakulta informačních technologií\\
Vysoké učení technické v~Brně}\\
\vspace{\stretch{0.382}}
{\LARGE
Typografie a publikování\,–\,2. projekt\\
Sazba dokumentů a matematických výrazů\\}
\vspace{\stretch{0.618}}
\end{center}
{\Large 2021 \hfill
Aleksandr Verevkin (xverev00)}

\end{titlepage}

\section*{Úvod}

V~této úloze si vyzkoušíme sazbu titulní strany, matematic\-kých vzorců, prostředí a dalších textových struktur obvyklých pro technicky zaměřené texty (například rovnice (\ref{eq1})
nebo Definice \ref{def1} na straně \pageref{def1}). Rovněž si vyzkoušíme používání odkazů \verb|\ref| a \verb|\pageref|.

Na titulní straně je využito sázení nadpisu podle optického středu s~využitím zlatého řezu. Tento postup byl probírán na přednášce. Dále je použito odřádkování se
zadanou relativní velikostí 0.4\,em a 0.3\,em.

V~případě, že budete potřebovat vyjádřit matematickou
konstrukci nebo symbol a nebude se Vám dařit jej nalézt
v~samotném \LaTeX u, doporučuji prostudovat možnosti balíku maker \AmS-\LaTeX.

\section{Matematický text}

Nejprve se podíváme na sázení matematických symbolů
a~výrazů v~plynulém textu včetně sazby definic a vět s~využitím balíku \verb|amsthm|. Rovněž použijeme poznámku pod čarou s~použitím příkazu \verb|\footnote|. Někdy je vhodné
použít konstrukci \verb|\mbox{}|, která říká, že text nemá být
zalomen.

\begin{definition}
\label{def1} Rozšířený zásobníkový automat \emph{(RZA) je definován jako sedmice tvaru $A = (Q, \Sigma, \Gamma, \delta , q_0, Z_0, F)$,
kde:}
\renewcommand\labelitemi{$\bullet$}
\begin{itemize}
    \item \emph{$Q$ je konečná množina} vnitřních (řídicích) stavů,
    \item \emph{$\Sigma$ je konečná} vstupní abeceda,
    \item \emph{$\Gamma$ je konečná} zásobníková abeceda,
    \item \emph{$\delta$ je} přechodová funkce $Q\times(\Sigma\cup\{\epsilon\})\times\Gamma^{\ast}\rightarrow2^{Q\times\Gamma^{\ast}}$,
    \item $q_0 \in Q$ \emph{je} počáteční stav, $Z_0 \in \Gamma$ \emph{je} startovací symbol
zásobníku \emph{a $F \subseteq Q$ je množina} koncových stavů.
\end{itemize}

Nechť $P = (Q, \Sigma, \Gamma, \delta, q_0, Z_0, F)$ je rozšířený zásobníkový automat. \emph{Konfigurací} nazveme trojici $(q, w, \alpha) \in Q \times \Sigma^{\ast} \times \Gamma^{\ast}$,
kde $q$ je aktuální stav vnitřního řízení,
$w$ je dosud nezpracovaná část vstupního řetězce a
$\alpha = Z_{i_1} Z_{i_2}\dots Z_{i_k}$ je obsah zásobníku\footnote{$Z_{i_1}$ je vrchol zásobníku}.
\end{definition}

\subsection{Podsekce obsahující větu a odkaz}
\begin{definition}
\label{def2} Řetězec $w$ nad abecedou $\Sigma$ je přijat RZA
\emph{A~jestliže} $(q_0, w, Z_0) \underset{A}{\overset{\ast}{\vdash}} (q_F , \epsilon, \gamma)$ \emph{pro nějaké} $\gamma \in \Gamma^{\ast}$ \emph{a} 
$q_F \in F$. \emph{Množinu} $L(A) = \{w \mid w$ \emph{je přijat RZA} $A$\} $\subseteq$ 
$\Sigma^{\ast}$~\emph{nazýváme} jazyk přijímaný RZA $A$.
\end{definition}
Nyní si vyzkoušíme sazbu vět a důkazů opět s~použitím
balíku \verb|amsthm|.


\begin{theorem}
Třída jazyků, které jsou přijímány ZA, odpovídá \emph{bezkontextovým jazykům}.
\end{theorem}

\begin{proof}
V~důkaze vyjdeme z~Definice \ref{def1} a \ref{def2}.
\end{proof}

\section{Rovnice a odkazy}

Složitější matematické formulace sázíme mimo plynulý
text. Lze umístit několik výrazů na jeden řádek, ale pak je
třeba tyto vhodně oddělit, například příkazem \verb|\quad|.

$$
\sqrt[i]{x^3_i} \quad
\text{kde $x_i$ je $i$-té sudé číslo splňující}
\quad x^{x^{i^{2}}_i+2}_i \leq y^{x^{4}_i}_i
$$

V~rovnici (\ref{eq1}) jsou využity tři typy závorek s~různou
explicitně definovanou velikostí.

\begin{eqnarray}
		\label{eq1} x & = & \bigg[ \Big\{\big[a + b\big] * c\Big\}^d \oplus 2 \bigg]^{3/2}\\
		y & = & \lim_{x\to\infty} \frac{\frac{1}{\log_{10} x}}{\sin^2x + \cos^2x} \nonumber
\end{eqnarray}

V~této větě vidíme, jak vypadá implicitní vysázení limity 
$\lim_{n\to\infty} f(n)$ v~normálním odstavci textu. Podobně
je to i s~dalšími symboly jako $\prod^n_{i=1} 2^i$ či $\bigcap_{A\in\mathcal{B}} A$. V~případě
vzorců $\lim\limits_{n\to\infty} f(n)$ a $\underset{i=1}{\overset{n}{\prod}} 2^i$ jsme si vynutili méně
úspornou sazbu příkazem \verb|\limits|.

\begin{eqnarray}
    \int^a_b g(x) \, \mathrm{d}x & = & - \int\limits^b_a f(x) \, \mathrm{d}x
\end{eqnarray}

\section{Matice}

Pro sázení matic se velmi často používá prostředí \verb|array|
a závorky (\verb|\left|, \verb|\right|).

$$
		\left(
		\begin{array}{ccc}
			a - b & \widehat{\xi + \omega} & \pi \\
			\Vec{\mathbf{a}} & \overleftrightarrow{AC} & \hat{\beta}\\
		\end{array}
		\right)
		= 1 \Longleftrightarrow \mathcal{Q} = \mathbb{R}
$$

$$
		\mathbf{A} =
		\left\|
		\begin{array}{cccc}
			a_{11} & a_{12} & \ldots & a_{1n} \\
			a_{21} & a_{22} & \ldots & a_{2n} \\
			\vdots & \vdots & \ddots & \vdots \\
			a_{m1} & a_{m2} & \ldots & a_{mn}
		\end{array}
		\right\|
		=
		\left|
		\begin{array}{cc}
			t & u~\\
			v~& w
		\end{array}
		\right|
		= tw - uv
$$

Prostředí \verb|array| lze úspěšně využít i jinde.

$$
		\binom{n}{k} =
		\left\{
		\begin{array}{cl}
			0 & \text{pro } k\ < 0 \text{ nebo } k\ > n \\
			\frac{n!}{k! (n - k)!} & \text{pro } 0 \leq k\ \leq n.
		\end{array}
		\right.
$$

\end{document}
